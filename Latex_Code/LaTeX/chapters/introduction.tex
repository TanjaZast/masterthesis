\chapter{Introduction}

In recent years, wearable technology has revolutionized health monitoring. Devices such as the Apple Watch, iPhone and Samsung smartphones play an important role in recording and analyzing health data. But they are also of great interest in the field of physical examinations ~\cite{healthtechmagazineLatestTrends}.

In addition to physical examinations, wearable technology also has a significant impact on the way athletes train and monitor their progress in sports and fitness. For example, the advanced sensors in these devices can track heart rate, oxygen saturation, sleep patterns and physical activity levels with high precision. This detailed monitoring allows athletes and fitness enthusiasts to gain insights into their performance, recovery and overall health ~\cite{doi:10.1177/1941738115616917}.

It was the integration of artificial intelligence (AI) into wearable technology, that has further enhanced its capabilities. AI algorithms can analyze the vast amounts of data collected by these devices to provide personalized feedback and recommendations. For example, AI can help predict potential injuries. This real-time analysis and feedback allows users to make informed decisions about their health routines ~\cite{s22186920}.

The combination of wearable technology and AI not only helps with personal health monitoring, but also opens up new possibilities for remote treatment. One example is the 6-Minute Walk Test (6MWT), which is often used to assess functional performance and predict cardiovascular and respiratory diseases. In this paper, the 6MWT is particularly emphasized as the University of Würzburg used this test for data collection with the Apple Watch Ultra. The data obtained from this test is directly incorporated into the study and is therefore an essential part of the methodology and analysis of the thesis. Patients can perform tests such as the 6MWT at home, and the data can be analyzed and shared with healthcare professionals in real time. This advance can be particularly beneficial in the treatment of chronic diseases, where continuous monitoring and timely interventions are crucial ~\cite{s22020581}.

This thesis will focus on investigating the efficiency of machine learning (ML) methods when using data collected exclusively from the Apple Watch Ultra, particularly in predicting health metrics. The key research questions guiding this study are: What data provided by the Apple Watch Ultra related to the 6MWT can be effectively utilized in ML models? Can we achieve reliable and accurate results with ML applications using only the data measured by the Apple Watch Ultra? Is the inclusion of non-Apple Watch Ultra data improving the performance and accuracy of ML models? By answering these questions, this research aims to identify the strengths and limitations of using data from wearable technologies for ML and ultimately optimize the integration of these technologies for improved health outcomes.

\section{Motivation}

The motivation behind this work is to investigate the accuracy of ML methods and clinical relevance of data from smartwatches such as the Apple Watch Ultra, particularly during the 6MWT. Many fitness enthusiasts already rely on data from these devices, but questions remain about their data reliability. Particularly in medical facilities, but also in clinical settings, this data reliability is fundamental ~\cite{b58999084d4b43b8bea456c509edf858}.

The aim of this thesis is to utilize the advanced sensors and data collection capabilities of the Apple Watch Ultra to train and test ML models. These models will predict health outcomes and identify potential health risks based on 6MWT data. The main goal is to evaluate the accuracy of the Apple Watch Ultra data and determine its utility for clinical applications.

This work builds on a detailed protocol and utilizes a dataset collected during a study involving the 6MWT. The focus is on integrating data from the Apple Watch Ultra with ML techniques to enhance health monitoring. By incorporating additional physiological and demographic variables such as blood pressure, sex and age, the predictive power and model performance will be improved.

This research also aims to close the gap in the clinical applicability of data from wearable technologies. It will collect and process data from the Apple Watch Ultra during the 6MWT to ensure it is ready for ML analysis. Both supervised and unsupervised ML models will be developed to analyze heart rates, running distances and other indicators during the 6MWT.

\section{Problem Statement and Research Question}

The aim of this research is to verify the accuracy and functionality of the Apple Watch Ultra during the 6MWT. By using ML, we hope to improve health predictions and better assess potential risks. The study addresses the research question of what data the Apple Watch Ultra provides and how this data can be used effectively in the field of supervised and unsupervised learning.

\section{Structure}

The paper is structured to systematically address the key challenges. The introductory chapter outlines the aims of the study, the problem statement and the general structure.

The second chapter contains the medical and technical background to which reference is made in the remainder of the work.

Chapter three, 'Related Work' [~\ref{cha:relatedwork}] provides an overview of the existing literature on ML in health monitoring, focusing on wearable technology and the 6MWT.

Chapter four, 'Methodology' [~\ref{cha:methods}] explains the data collection, pre-processing steps and the ML models used for analysis.

Chapters five and six, 'Example Dataset' [~\ref{cha:exampleDataset}] and 'Analyzing the Study Dataset' [~\ref{cha:studyDataSet}] discuss the structure of the collected dataset and present the initial results of the exploratory data analysis. This leads on to chapter seven, 'Implementation and Results' [~\ref{cha:results}], which covers the practical implementation of various ML models, such as linear regression, random forests and deep learning techniques. This chapter also presents the results of the model evaluations, including performance metrics and comparisons with existing methods.

Chapter eight, 'Improving Model Performance with additional Health Data' [~\ref{cha:resultsaddFeatures}] deals with specific improvements and experiments. It includes sections on distance measurement accuracy and analyzing blood pressure data, adding more measures to the dataset. Each section addresses different ways to improve data analysis and prediction accuracy.

Finally, chapter nine, 'Discussion and Further Work' [~\ref{cha:discussion}], delves into the implications of the findings, addresses the limitations of the study, and proposes suggestions for future research. Additionally, it outlines preliminary experiments and approaches for further work, including the use of a larger dataset to enhance the robustness and generalizability of the results.